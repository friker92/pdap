\documentclass[10pt]{beamer}

\usetheme{m}

\usepackage{times}
\usepackage{tikz}
\usepackage{amsmath}
\usepackage{fancyvrb}
\usepackage{xspace}

\usepackage{booktabs}
\usepackage[scale=2]{ccicons}

% \usepgfplotslibrary{dateplot}

\usepackage[latin1]{inputenc}
\usepackage[spanish]{babel}

\usepackage{suffix}
\usepackage{listings}
\usepackage{pgfplots}
\usepackage{color}
\definecolor{lightgray}{rgb}{.9,.9,.9}
\definecolor{darkgray}{rgb}{.4,.4,.4}
\definecolor{purple}{rgb}{0.65, 0.12, 0.82}


\title{Javascript Funcional}
\subtitle{}
\date{}
\author{Jes�s Javier Dom�nech Arellano}
\institute{27 Enero 2016}

\titlegraphic{\hfill\includegraphics[scale=0.29]{images/logo.png}}

\definecolor{bgg}{HTML}{FBFBFB}
\def\gcolor{bgg}    % while presenting
% \def\gcolor{black} % while developing

\def\tikzpicdim{
  \draw[step=0.1cm, color=\gcolor] (0,0) grid (12,7);
  \draw[step=1cm, color=\gcolor] (0,0) grid (12,7);
}

\let\tikzpicdimlarge\tikzpicdim

\def\myurl{\hfil\penalty 100 \hfilneg \hbox}

\metroset{titleformat=regular}
\metroset{inner/sectiontitleformat=regular}
\metroset{outer/frametitleformat=regular}
\metroset{block=fill}

\newcommand\sectionDark[1]{{\metroset{background=dark} \section{#1} }}
\WithSuffix\newcommand\sectionDark*[1]{{\metroset{background=dark} \section*{#1}}}

% == Estilos propios de lstlisting

\lstdefinelanguage{JavaScript}{
  keywords={typeof, new, true, false, catch, function, return, null,
    catch, switch, var, if, in, while, do, for, default, else, case, break},
  keywordstyle=\color{blue},%\bfseries,
  ndkeywords={class, export, boolean, throw, implements, import, this},
  ndkeywordstyle=\color{darkgray}\bfseries,
  identifierstyle=\color{black},
  sensitive=false,
  comment=[l]{//},
  morecomment=[s]{/*}{*/},
  commentstyle=\color{purple}\ttfamily,
  stringstyle=\color{red}\ttfamily,
  morestring=[b]',
  morestring=[b]"
}

\lstset{
   language=JavaScript,
  % backgroundcolor=\color{yellow!20},%
  basicstyle=\scriptsize\ttfamily,% breaklines=true,
  numbers=left, numberstyle=\tiny, stepnumber=1, numbersep=5pt,%
  frame=single, showstringspaces=false,
   showspaces=false%
}%

% Add your keywords here, and have this in a separate file
% and include it in your preamble
% \lstset{emph={%  
% procedure, end, to, do, pragma, omp, parallel, for, private%
% },emphstyle={\color{black}\bfseries\underline}%
% }%

\begin{document}

\maketitle

%% ======= INDICE ========================================
\begin{frame}
  \frametitle{�ndice}
  \setbeamertemplate{section in toc}[sections numbered]
  \tableofcontents[hideallsubsections]
\end{frame}

\sectionDark{�Por qu� Javascript?}

\begin{frame}[fragile,fragile]
  \frametitle{�Por qu� Javascript (JS)?}
  \begin{tikzpicture}
    \tikzpicdimlarge
    \only<1->{\node[] (def) at (0.5,6) {
        \begin{minipage}{0.9\textwidth}
          Este lenguaje posee las siguientes caracter�sticas:
          \begin{itemize}
          \item Es dinamicamente tipado.
          \item Esta orientado a objetos (prototype).
          \item Funcional.
          \item De prop�sito general.
          \end{itemize}
        \end{minipage}
      };}
    
    \only<2->{\node[] (def) at (0.5,2) {
        \begin{minipage}{0.9\textwidth}
          Adem�s, est� presente en:
          \begin{itemize}
          \item Aplicaciones m�viles.
          \item Sitios web.
          \item Servidores web.
          \item Aplicaciones de escritorio.
          \item Bases de datos.
          \end{itemize}
        \end{minipage}
      };}
  \end{tikzpicture}
\end{frame}

\sectionDark{JavaScript en 3 minutos}

\begin{frame}[fragile,fragile]
  \frametitle{JS en 3 min}
  \alert{Expresiones:}
\begin{lstlisting}
// N�meros siempre son doubles
3; // -> 3
1.5; // -> 1.5 
((10+4*0.5-2)/2)%3; // -> 1?

// N�meros no reales
Infinity;
-Infinity;
NaN; 

// Booleanos
true;
false;

// Strings
'hola mundo!';
"hola mundo!";

//Sin valor;
null; // no tiene valor aposta
undefined; // no tiene un valor en este �mbito todav�a
\end{lstlisting}
\end{frame}

\begin{frame}[fragile,fragile]
  \frametitle{JS en 3 min}

  \alert{Operadores:}
\begin{lstlisting}
// + - * / %

!true; // negacion
1 == 1; // < > <= >=

"Hola " + 'mundo!'; //contatenar
//tambien puedes comparar strings 
"a" < 'b'; // -> true

//ATENCION
"5" == 5; // -> true NO IMPORTA EL TIPO!
"5" === 5; // -> false 
\end{lstlisting}
  \alert{Operador Ternario:}
\begin{lstlisting}
 condicion ? instruccionCierto : instruccionFalso;
\end{lstlisting}
\end{frame}

\begin{frame}[fragile,fragile]
  \frametitle{JS en 3 min}
  \alert{Variables:}
\begin{lstlisting}
var nombreDeVariable = 2;

variableSinPoneVar = 22; // ahora es una variable global

var variableSinValor; // -> ahora vale undefined
\end{lstlisting}
  \alert{Colecciones:}
\begin{lstlisting}
var miArray = [ "hola", 22, true, null, 2.2];

miArray[2]; // -> true

miArray.push("otro");
miArray.length; //-> 6

// diccionarios == Objetos
var miObjeto = {clave:"valor", "clave 2": [1,"valor"]};

miObjeto.clave; // -> "valor"
miObjeto["clave"]; // -> "valor"
miObjeto.clave3 = 4; 
miObjeto.clave4; // -> undefined
\end{lstlisting}
\end{frame}

\begin{frame}[fragile,fragile,fragile]
  \frametitle{JS en 3 min}
  \alert{Estructuras de Control:}
\begin{lstlisting}
if ( condicion ) {
} else if ( condicion2 ) {
} else {
}

while ( condicion ) {
}

do {
}while ( condicion )

for ( init; condicion; iteracion) {
}

for ( var key in variable ){  // variable puede ser un Objeto o Array
}

switch ( variable ) {
  case Valor1:
    break;
  default:
   break;
}
\end{lstlisting}
\end{frame}
\begin{frame}[fragile,fragile,fragile]
  \frametitle{JS en 3 min}
  \alert{Funciones:}
\begin{lstlisting}
   function (param1, param2, ...){
      instrucci�n1;
      instrucci�n2;
      ...
      return instrucci�nfinal;
   }

   function Nombre( ... ){ ... }

   var Nombre2 = function ( ... ) { ... };

\end{lstlisting}
  \alert{�Punto y coma?:} No es necesario ponerlo siempre. Pero,
  \alert{�cuidado!}
\begin{lstlisting}
   function foo (parametro){
      return   // -> se a�ade un ; automaticamente 
      {
         clave: parametro
      }
   }

   foo(10) // -> undefined

\end{lstlisting}

\end{frame}

\sectionDark{JavaScript Funcional}

\begin{frame}[fragile,fragile,fragile]
  \frametitle{JS Funcional}
  \alert{$\lambda$ expresiones:}
\begin{lstlisting}
(param1, param2, ...) => expresi�n;

(param1, param2, ...) => {
                           predicado1;
                           predicado2;
                           ...
                           return predicadofinal;
                         }
\end{lstlisting}
  \alert{Ejemplo:}
\begin{lstlisting}
    var multiplicar = (a,b) => a * b;

    var apply = (opt, a, b) => opt(a,b);
    apply(multiplicar, 4, 8); // -> 32
\end{lstlisting}

\end{frame}

\begin{frame}[fragile,fragile,fragile]
  \frametitle{JS Funcional}
  \begin{tikzpicture}
    \tikzpicdimlarge
    \begin{onlyenv}<1->
      \node at (0.5,6) {
        \begin{minipage}{0.9\textwidth}
          \alert{Erlang:}
\begin{lstlisting}
condici�n1 -> 
              instrucciones1;
condici�n2 -> 
              instrucciones2;
...
condici�nN ->
              instruccionesN;
true       ->
              instruccionesDefault;
\end{lstlisting}
        \end{minipage}
      };
    \end{onlyenv}

    \begin{onlyenv}<2->
      \node at (0.5,2.4) {
        \begin{minipage}{0.9\textwidth}
          \alert{Javascript:}
\begin{lstlisting}
 condici�n1 ? instrucciones1 : 
 condici�n2 ? instrucciones2 : 
 ...
 condici�nN ? instruccionesN : 
              instruccionesDefault;
\end{lstlisting}
        \end{minipage}
      };
    \end{onlyenv}    

  \end{tikzpicture}
\end{frame}

\begin{frame}
  \frametitle{JS Funcional}
  UnderScore y Lodash son dos librer�as que definen un entorno contenido
  en la variable ``\_''. Nos proporciona diferentes m�todos:

  \begin{itemize}
  \item \emph{forEach}: recibe una lista y una funci�n, se ejecuta la
    funci�n una vez por cada elemento de la lista.
  \item \emph{map}: recibe una lista y una funci�n, devuelve el
    resultado de ejecutar la funci�n a cada elemento de la lista.
  \item \emph{filter}: recibe una lista y una funci�n, devuelve los
    valores de la lista que dan cierto en la funci�n pasada.
  \item \emph{reduce}: recibe una lista, una funci�n y un elemento
    base. La funci�n recibe un acumulador y un elemento de la
    lista. Devuelve el valor final de la acumulaci�n.
  \item \emph{curry}: recibe una funci�n con N argumentos y la
    currifica.
  \end{itemize}

\end{frame}

\begin{frame}
  \frametitle{JS Funcional}
  \alert{�M�s?}
  \begin{itemize}
  \item \emph{Wrappers}
  \item \emph{Funtores}
  \item \emph{IO Monads}
  \item \emph{Acceso al DOM}
  \end{itemize}
\end{frame}



\sectionDark{Ejemplos}

\begin{frame}[fragile,fragile,fragile]
  \frametitle{forEach}
  \begin{tikzpicture}
    \tikzpicdimlarge
    \begin{onlyenv}<1->
      \node at (0.5,6) {
        \begin{minipage}{0.9\textwidth}
\begin{lstlisting}
function forEach(array, action) {
  for (var i = 0; i < array.length; i++)
    action(array[i]);
}
\end{lstlisting}
        \end{minipage}
      };
    \end{onlyenv}

    \begin{onlyenv}<2->
      \node at (0.5,2.4) {
        \begin{minipage}{0.9\textwidth}
          \alert{Ejemplo:}
\begin{lstlisting}
function sum(numbers) {
  var total = 0;
  forEach(numbers, function (number) {
    total += number;
  });
  return total;
}

console.log(sum([1, 10, 100])); // -> 111
\end{lstlisting}
        \end{minipage}
      };
    \end{onlyenv}    

  \end{tikzpicture}
\end{frame}

\begin{frame}[fragile,fragile,fragile]
  \frametitle{reduce}
  \begin{tikzpicture}
    \tikzpicdimlarge
    \begin{onlyenv}<1->
      \node at (0.5,6) {
        \begin{minipage}{0.9\textwidth}
\begin{lstlisting}
function reduce(action, base, array) {
  forEach(array, function (element) {
    base = action(base, element);
  });
  return base;
}
\end{lstlisting}
        \end{minipage}
      };
    \end{onlyenv}

    \begin{onlyenv}<2->
      \node at (0.5,2.4) {
        \begin{minipage}{0.9\textwidth}
          \alert{Ejemplo:}
\begin{lstlisting}
function add(a, b) {
  return a + b;
}

function sum2(numbers) {
  return reduce(add, 0, numbers);
}

console.log(sum2([1, 10, 100])); // -> 111
\end{lstlisting}
        \end{minipage}
      };
    \end{onlyenv}    

  \end{tikzpicture}
\end{frame}

\begin{frame}[fragile,fragile,fragile]
  \frametitle{map}
  \begin{tikzpicture}
    \tikzpicdimlarge
    \begin{onlyenv}<1->
      \node at (0.5,6) {
        \begin{minipage}{0.9\textwidth}
\begin{lstlisting}
function map(action, array) {
  var result = [];
  forEach(array, function (element) {
    result.push(action(element));
  });
  return result;
}

\end{lstlisting}
        \end{minipage}
      };
    \end{onlyenv}

    \begin{onlyenv}<2->
      \node at (0.5,2.4) {
        \begin{minipage}{0.9\textwidth}
          \alert{Ejemplo:}
\begin{lstlisting}
console.log(map(Math.round,[0.1,9.8,Math.PI])); // -> [0, 10, 3]
\end{lstlisting}
        \end{minipage}
      };
    \end{onlyenv}    

  \end{tikzpicture}
\end{frame}

\begin{frame}[fragile,fragile,fragile]
  \frametitle{filter}
  \begin{tikzpicture}
    \tikzpicdimlarge
    \begin{onlyenv}<1->
      \node at (0.5,6) {
        \begin{minipage}{0.9\textwidth}
\begin{lstlisting}
function filter(func, array) {
  var result = [];
  forEach(array, function (element) {
    if (func(element))
      result.push(element);
  });
  return result;
}

\end{lstlisting}
        \end{minipage}
      };
    \end{onlyenv}

    \begin{onlyenv}<2->
      \node at (0.5,2.4) {
        \begin{minipage}{0.9\textwidth}
          \alert{Ejemplo:}
\begin{lstlisting}
function lt(n){
  return function(a){
    return a < n;
  }
}

console.log(filter(lt(5), [3, 9.89, 4.8])); // -> [3, 4.8]
\end{lstlisting}
        \end{minipage}
      };
    \end{onlyenv}    

  \end{tikzpicture}
\end{frame}




\sectionDark{Bibliograf�a}
% ======= BIBLIOGRAF�A =======
\begin{frame}
  \frametitle{Bibliograf�a}
  \begin{enumerate}
  \item \url{https://dzone.com/storage/assets/379008-rc217-functionalprogramming.pdf}
  \item \url{http://bonsaiden.github.io/JavaScript-Garden}
  \item \url{http://eloquentjavascript.net/1st_edition/chapter6.html}
  \end{enumerate}
\end{frame}

\sectionDark*{ � ? }

\end{document}

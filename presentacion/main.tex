\documentclass[10pt]{beamer}

\usetheme{m}

\usepackage{times}
\usepackage{tikz}
\usepackage{amsmath}
\usepackage{fancyvrb}
\usepackage{xspace}

\usepackage{booktabs}
\usepackage[scale=2]{ccicons}

%\usepackage{pgfplots}
%\usepgfplotslibrary{dateplot}

\usepackage[latin1]{inputenc}
\usepackage[spanish]{babel}

\usepackage{listings}
\usepackage{pgfplots}


\title{Javascript Funcional}
\subtitle{}
\date{}
\author{Jes�s Javier Dom�nech Arellano}
\institute{27 Enero 2016}

\titlegraphic{\hfill\includegraphics[scale=0.29]{images/logo.png}}

\definecolor{bgg}{HTML}{FBFBFB}
\def\gcolor{bgg}    % while presenting
%\def\gcolor{black} % while developing

\def\tikzpicdim{
  \draw[step=0.1cm, color=\gcolor] (0,-1) grid (12,7);
  \draw[step=1cm, color=\gcolor] (0,-1) grid (12,7);
}

\let\tikzpicdimlarge\tikzpicdim

\def\myurl{\hfil\penalty 100 \hfilneg \hbox}

\metroset{titleformat=regular}
\metroset{inner/sectiontitleformat=regular}
\metroset{outer/frametitleformat=regular}
\metroset{block=fill}
\lstset{basicstyle=\footnotesize\ttfamily,breaklines=true, postbreak={\mbox{$\hookrightarrow\space$}}, language=Mathematica}

\usepackage[framemethod=TikZ]{mdframed}
\mdfdefinestyle{lstlisting}{%
 innertopmargin=5pt,
 middlelinewidth=1pt,
 outerlinewidth=9pt,outerlinecolor=white,
 innerleftmargin=10pt,
 innerrightmargin=10pt,
 leftmargin=-9pt,rightmargin=-9pt,
 skipabove=\topskip,
 skipbelow=\topskip,
 roundcorner=5pt,
 singleextra={\node[draw, fill=white,anchor=west, xshift=10pt+1pt,font=\bfseries] at (O|-P) {Code Listings};},
 firstextra={\node[draw, fill=white,anchor=west, xshift=10pt+1pt,font=\bfseries] at (O|-P) {Code Listings};}
}

\newcommand\sectionDark[1]{{\metroset{background=dark} \section{#1} }}

%== Estilos propios de lstlisting
\lstset{%
  %backgroundcolor=\color{yellow!20},%
    basicstyle=\tiny\ttfamily,breaklines=true,
    numbers=left, numberstyle=\tiny, stepnumber=1, numbersep=5pt,%
    frame=single%
    }%

% Add your keywords here, and have this in a separate file
% and include it in your preamble
%\lstset{emph={%  
%    procedure, end, to, do, pragma, omp, parallel, for, private%
%    },emphstyle={\color{black}\bfseries\underline}%
%}%



\begin{document}

\maketitle



%%======= INDICE ========================================
%\begin{frame}
%  \frametitle{�ndice}
%  \setbeamertemplate{section in toc}[sections numbered]
%  \tableofcontents[hideallsubsections]
%\end{frame}

%\section{Introduction} %=== Esrto genera una p�gina de secci�n.



\begin{frame}[fragile,fragile]
  \frametitle{Javascript}
  \begin{tikzpicture}
    \tikzpicdimlarge
    \only<1->{\node[] (def) at (0.5,6) {
        \begin{minipage}{0.9\textwidth}
          Este lenguaje posee las siguientes caracter�sticas:
          \begin{itemize}
          \item Es dinamicamente tipado.
          \item Esta orientado a objetos (prototype).
          \item Funcional.
          \item De prop�sito general.
          \end{itemize}
        \end{minipage}
      };}
    
    \only<2->{\node[] (def) at (0.5,2) {
        \begin{minipage}{0.9\textwidth}
          Adem�s, est� presente en:
          \begin{itemize}
          \item Aplicaciones m�viles.
          \item Sitios web.
          \item Servidores web.
          \item Aplicaciones de escritorio.
          \item Bases de datos.
          \end{itemize}
        \end{minipage}
      };}
  \end{tikzpicture}
\end{frame}


\begin{frame}[fragile,fragile,fragile]
  \frametitle{}
    \begin{tikzpicture}
    \tikzpicdimlarge
    \begin{onlyenv}<1->
    \node at (0.5,6) {
      \begin{minipage}{0.9\textwidth}
\begin{lstlisting}
    (param1, param2, ...) => expresi�n;
\end{lstlisting}

       
        \end{minipage}
      };
    \end{onlyenv}
    
    \only<2->{\node at (0.5,2) {
        \begin{minipage}{0.9\textwidth}
          Adem�s, est� presente en:
          \begin{itemize}
          \item Aplicaciones m�viles.
          \item Sitios web.
          \item Servidores web.
          \item Aplicaciones de escritorio.
          \item Bases de datos.
          \end{itemize}
        \end{minipage}
      };}
  \end{tikzpicture}
\end{frame}

\begin{frame}
  \frametitle{Pruebas realizadas}

  \alert{Pruebas realizadas:}
  \begin{itemize}
  \item Ejecuci�n de Cholesky \textbf{secuencial} (gcc \& icc).
  \item Ejecuci�n de Cholesky usando \textbf{16 pthreads} (gcc \& icc).
  \item Ejecuci�n de Cholesky utilizando \textbf{OpenMP - gcc}.
  \item Ejecuci�n de Cholesky utilizando \textbf{OpenMP - Intel}.
  \end{itemize}

  \alert{Caracter�sticas de las pruebas:}
  \begin{itemize}
  \item Matriz 4096x4096 elementos.
  \item 16 hilos de ejecuci�n.
  \end{itemize}
  \alert{Caracter�sticas de la m�quina:}
  \begin{itemize}
  \item Intel Xeon E5530 - 2.4 GHz
  \item 48 GB RAM
  \item Debian Jessie, Kernel 3.14-1-amd64
  \item gcc v4.8.3
  \item icc v14.0.3
  \end{itemize}

\end{frame}

\begin{frame}
  \frametitle{Resultados}
  
  \begin{figure}
    \centering
    \begin{tikzpicture}
      \begin{axis}[
        mbarplot,
        xlabel={Tipo de ejecuci�n},
        ylabel={Tiempo (s)},
        symbolic x coords={secuencial,pthread,OpenMP},
        xtick=data,
        nodes near coords,
        nodes near coords align={vertical},
        ]
        \addplot coordinates {(secuencial,43.17) (pthread,27.70) (OpenMP,12.76)};
        \addplot coordinates {(secuencial,20.97) (pthread,10.81) (OpenMP,5.58)};
        \legend{gcc, icc}
      \end{axis}
    \end{tikzpicture}
  \end{figure}


\end{frame}

%======= BIBLIOGRAF�A =======
\begin{frame}
  \frametitle{Bibliograf�a}
  
  \begin{enumerate}
  \item \url{https://pm.bsc.es/}
  \item \url{https://software.intel.com/en-us/c-compilers}
  \item \url{https://gcc.gnu.org/}
  \item \url{https://es.wikipedia.org/wiki/Factorizaci\%C3\%B3n_de_Cholesky}
  \end{enumerate}
\end{frame}

\end{document}
